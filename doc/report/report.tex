\documentclass{article}
\usepackage[utf8]{inputenc}
\usepackage{amsmath}
\usepackage{amsfonts}
\usepackage{amssymb}
\usepackage[cm]{fullpage}
\usepackage{hyperref}
\usepackage{enumitem}
\usepackage{array}
\usepackage{graphicx}
\usepackage{caption}
\usepackage{subcaption}
\usepackage{float}
\usepackage{moreverb}
\setdescription{labelindent=\parindent}
%%%%%%%%%%%%%%%%%%%%%%%%%%%%%%%%%%%%%%%%%
% Code Snippet
% LaTeX Template
% Version 1.0 (14/2/13)
%
% This template has been downloaded from:
% http://www.LaTeXTemplates.com
%
% Original author:
% Velimir Gayevskiy (vel@latextemplates.com)
%
% License:
% CC BY-NC-SA 3.0 (http://creativecommons.org/licenses/by-nc-sa/3.0/)
%
%%%%%%%%%%%%%%%%%%%%%%%%%%%%%%%%%%%%%%%%%
%----------------------------------------------------------------------------------------

\usepackage{listings} % Required for inserting code snippets
\usepackage[usenames,dvipsnames]{color} % Required for specifying custom colors and referring to colors by name

\definecolor{DarkGreen}{rgb}{0.0,0.4,0.0} % Comment color
\definecolor{highlight}{RGB}{255,251,204} % Code highlight color

\lstdefinestyle{Octave}{ % Define a style for your code snippet, multiple definitions can be made if, for example, you wish to insert multiple code snippets using different programming languages into one document
language=Octave, % Detects keywords, comments, strings, functions, etc for the language specified
%backgroundcolor=\color{highlight}, % Set the background color for the snippet - useful for highlighting
backgroundcolor=\color{white}, % Set the background color for the snippet
basicstyle=\footnotesize\ttfamily, % The default font size and style of the code
breakatwhitespace=false, % If true, only allows line breaks at white space
breaklines=true, % Automatic line breaking (prevents code from protruding outside the box)
captionpos=b, % Sets the caption position: b for bottom; t for top
commentstyle=\usefont{T1}{pcr}{m}{sl}\color{DarkGreen}, % Style of comments within the code - dark green courier font
deletekeywords={}, % If you want to delete any keywords from the current language separate them by commas
%escapeinside={\%}, % This allows you to escape to LaTeX using the character in the bracket
firstnumber=1, % Line numbers begin at line 1
frame=single, % Frame around the code box, value can be: none, leftline, topline, bottomline, lines, single, shadowbox
frameround=tttt, % Rounds the corners of the frame for the top left, top right, bottom left and bottom right positions
keywordstyle=\color{Blue}\bf, % Functions are bold and blue
morekeywords={}, % Add any functions no included by default here separated by commas
numbers=left, % Location of line numbers, can take the values of: none, left, right
numbersep=10pt, % Distance of line numbers from the code box
numberstyle=\tiny\color{Gray}, % Style used for line numbers
rulecolor=\color{black}, % Frame border color
showstringspaces=false, % Don't put marks in string spaces
showtabs=false, % Display tabs in the code as lines
stepnumber=5, % The step distance between line numbers, i.e. how often will lines be numbered
stringstyle=\color{Purple}, % Strings are purple
tabsize=2, % Number of spaces per tab in the code
}
\lstdefinestyle{Python}{ % Define a style for your code snippet, multiple definitions can be made if, for example, you wish to insert multiple code snippets using different programming languages into one document
language=Python, % Detects keywords, comments, strings, functions, etc for the language specified
%backgroundcolor=\color{highlight}, % Set the background color for the snippet - useful for highlighting
backgroundcolor=\color{white}, % Set the background color for the snippet
basicstyle=\footnotesize\ttfamily, % The default font size and style of the code
breakatwhitespace=false, % If true, only allows line breaks at white space
breaklines=true, % Automatic line breaking (prevents code from protruding outside the box)
captionpos=b, % Sets the caption position: b for bottom; t for top
commentstyle=\usefont{T1}{pcr}{m}{sl}\color{DarkGreen}, % Style of comments within the code - dark green courier font
deletekeywords={}, % If you want to delete any keywords from the current language separate them by commas
%escapeinside={\%}, % This allows you to escape to LaTeX using the character in the bracket
firstnumber=1, % Line numbers begin at line 1
frame=single, % Frame around the code box, value can be: none, leftline, topline, bottomline, lines, single, shadowbox
frameround=tttt, % Rounds the corners of the frame for the top left, top right, bottom left and bottom right positions
keywordstyle=\color{Blue}\bf, % Functions are bold and blue
morekeywords={}, % Add any functions no included by default here separated by commas
numbers=left, % Location of line numbers, can take the values of: none, left, right
numbersep=10pt, % Distance of line numbers from the code box
numberstyle=\tiny\color{Gray}, % Style used for line numbers
rulecolor=\color{black}, % Frame border color
showstringspaces=false, % Don't put marks in string spaces
showtabs=false, % Display tabs in the code as lines
stepnumber=5, % The step distance between line numbers, i.e. how often will lines be numbered
stringstyle=\color{Purple}, % Strings are purple
tabsize=2, % Number of spaces per tab in the code
}

% Create a command to cleanly insert a snippet with the style above anywhere in the document
\newcommand{\insertcodeOctave}[2]{\begin{itemize}\item[]\lstinputlisting[caption=#2,label=#1,style=Octave]{#1}\end{itemize}} % The first argument is the script location/filename and the second is a caption for the listing
\newcommand{\insertcodePython}[2]{\begin{itemize}\item[]\lstinputlisting[caption=#2,label=#1,style=Python]{#1}\end{itemize}} % The first argument is the script location/filename and the second is a caption for the listing
% End Code Snippet
%%%%%%%%%%%%%%%%%%%%%%%%%%%%%%%%%%%%%%%%%
% Vector shortcuts
\newcommand*{\X}{\Vec{x}}
\newcommand*{\Y}{\Vec{y}}
\newcommand*{\Z}{\Vec{z}}
\newcommand*{\0}{\Vec{0}}
\newcommand*{\norm}[1]{\lVert#1\rVert_2}
% Greek shortcuts
\newcommand*{\al}{\alpha}
\newcommand*{\be}{\beta}
\newcommand*{\de}{\delta}
\newcommand*{\la}{\lambda}
\newcommand*{\om}{\omega}
\newcommand*{\ep}{\epsilon}
\begin{document}
\title
{\begin{flushleft}
\large
Patrick Pegus\\
Mini Project 2\\
\today\\
CMPSCI-689\\
Prof. Sridhar Mahadevan
\end{flushleft}}
\author{}
\date{}
\maketitle
\normalsize
\begin{enumerate}
	\item
		\begin{enumerate}
			\item
				The state variables at time $t$ are marginally independent because the observation at that time d-separates them.
				In other words, $P(S(t,1)|S(t,2)) = P(S(t,1))$ because $Y(t)$ is a collider node in their path.
				However, $P(S(t,1)|S(t,2)) \ne P(S(t,1))$ when $Y(t)$ then d-connects them.
				The state variables at time $t$ are conditionally independent of the past history of state variables given the state variables at $t-1$ because those given variables d-separate them from past states.
			\item
				To convert the factorial HMM to a regular HMM, collapse states $S(t,1),\dots S(t,M)$ to a single state $S(t)$ that has $K^M$ values, which is enough to represent all possible state combinations of the former states.
				Since the time complexity of the forward algorithm on an HMM is $O(L^2T)$ where $L$ is the number of state values, the complexity of the converted HMM is $O\left(\left(K^{M}\right)^2T\right)=O(K^{2M}T)$.
		\end{enumerate}
	\item
		\begin{enumerate}
			\item
				\begin{description}
					\item[Lagrange dual]
						\begin{align*}
							L(w,\xi,\al) &= \la\|w\|^2 +\sum_{i=1}^l\xi_i^2+\sum_{i=1}^l\al_i(y_i - \langle w, x_i \rangle - \xi_i) \\
							\max_\al &\left(L_D(\al) = \min_{w,\xi}L(w,\xi,\al)\right)
						\end{align*}
					\item[Assure 0 duality gap]
						\begin{align*}
							0 &= \frac{\de}{\de w}L(w,\xi,\al)
							= 2\la w - \sum_{i=1}^l\al_i x_i \iff
							w = \frac{1}{2\la}\sum_{i=1}^l\al_i x_i \\
							0 &= \frac{\de}{\de \xi_k}L(w,\xi,\al)
							= 2\xi_k-\al_k \iff \xi_k = \frac{\al_k}{2} \\
							L_D(\al)
							&= \la\| \frac{1}{2\la}\sum_{i=1}^l\al_i x_i \|^2 +\sum_{i=1}^l\left( \frac{\al_i}{2} \right)^2+\sum_{i=1}^l\al_i(y_i - \langle \frac{1}{2\la}\sum_{j=1}^l\al_j x_j, x_i \rangle - \frac{\al_i}{2}) \\
							&= \frac{1}{4\la}\| \sum_{i=1}^l\al_i x_i \|^2 +\sum_{i=1}^l \frac{\al_i^2}{4} +\sum_{i=1}^l\al_iy_i - \frac{1}{2\la}\sum_{i=1}^l\al_i\langle \sum_{j=1}^l\al_j x_j, x_i \rangle - \sum_{i=1}^l \frac{\al_i^2}{2} \\
							&= \frac{1}{4\la}\sum_{i=1}^l\al_i\langle \sum_{j=1}^l\al_j x_j, x_i \rangle -\sum_{i=1}^l \frac{\al_i^2}{4} +\sum_{i=1}^l\al_iy_i - \frac{1}{2\la}\sum_{i=1}^l\al_i\langle \sum_{j=1}^l\al_j x_j, x_i \rangle \\
							&= \sum_{i=1}^l\al_iy_i - \frac{1}{4\la}\sum_{i=1}^l\al_i\langle \sum_{j=1}^l\al_j x_j, x_i \rangle -\sum_{i=1}^l \frac{\al_i^2}{4} \\
							&= \sum_{i=1}^l\al_iy_i - \frac{1}{4\la}\sum_{i=1}^l\al_i \sum_{j=1}^l\al_j \langle x_j, x_i \rangle -\sum_{i=1}^l \frac{\al_i^2}{4} \\
						\end{align*}
				\end{description}
			\item Solution to kernel ridge regression occurs when $\frac{\de}{\de \al}L_D(\al)=0$.
				\begin{align*}
					0 &= \frac{\de}{\de \al}L_D(\al)
					= \ldots \text{working backwards, but can't figure out this step} \ldots
					= y -\frac{G\al}{2\la} - \frac{\al}{2} \\
					0 &= y - \left(G + \la I\right)\frac{\al}{2\la} \\
					\left(G + \la I\right)\frac{\al}{2\la} &= y \\
					\frac{\al}{2\la} &= \left(G + \la I\right)^{-1}y \\
					\al &= 2\la\left(G + \la I\right)^{-1}y \\
				\end{align*}
		\end{enumerate}
\end{enumerate}

%\bibliographystyle{abbrv}
%\bibliography{report}

\end{document}
